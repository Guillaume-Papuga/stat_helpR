% Options for packages loaded elsewhere
\PassOptionsToPackage{unicode}{hyperref}
\PassOptionsToPackage{hyphens}{url}
%
\documentclass[
]{book}
\usepackage{amsmath,amssymb}
\usepackage{iftex}
\ifPDFTeX
  \usepackage[T1]{fontenc}
  \usepackage[utf8]{inputenc}
  \usepackage{textcomp} % provide euro and other symbols
\else % if luatex or xetex
  \usepackage{unicode-math} % this also loads fontspec
  \defaultfontfeatures{Scale=MatchLowercase}
  \defaultfontfeatures[\rmfamily]{Ligatures=TeX,Scale=1}
\fi
\usepackage{lmodern}
\ifPDFTeX\else
  % xetex/luatex font selection
\fi
% Use upquote if available, for straight quotes in verbatim environments
\IfFileExists{upquote.sty}{\usepackage{upquote}}{}
\IfFileExists{microtype.sty}{% use microtype if available
  \usepackage[]{microtype}
  \UseMicrotypeSet[protrusion]{basicmath} % disable protrusion for tt fonts
}{}
\makeatletter
\@ifundefined{KOMAClassName}{% if non-KOMA class
  \IfFileExists{parskip.sty}{%
    \usepackage{parskip}
  }{% else
    \setlength{\parindent}{0pt}
    \setlength{\parskip}{6pt plus 2pt minus 1pt}}
}{% if KOMA class
  \KOMAoptions{parskip=half}}
\makeatother
\usepackage{xcolor}
\usepackage{longtable,booktabs,array}
\usepackage{calc} % for calculating minipage widths
% Correct order of tables after \paragraph or \subparagraph
\usepackage{etoolbox}
\makeatletter
\patchcmd\longtable{\par}{\if@noskipsec\mbox{}\fi\par}{}{}
\makeatother
% Allow footnotes in longtable head/foot
\IfFileExists{footnotehyper.sty}{\usepackage{footnotehyper}}{\usepackage{footnote}}
\makesavenoteenv{longtable}
\usepackage{graphicx}
\makeatletter
\def\maxwidth{\ifdim\Gin@nat@width>\linewidth\linewidth\else\Gin@nat@width\fi}
\def\maxheight{\ifdim\Gin@nat@height>\textheight\textheight\else\Gin@nat@height\fi}
\makeatother
% Scale images if necessary, so that they will not overflow the page
% margins by default, and it is still possible to overwrite the defaults
% using explicit options in \includegraphics[width, height, ...]{}
\setkeys{Gin}{width=\maxwidth,height=\maxheight,keepaspectratio}
% Set default figure placement to htbp
\makeatletter
\def\fps@figure{htbp}
\makeatother
\setlength{\emergencystretch}{3em} % prevent overfull lines
\providecommand{\tightlist}{%
  \setlength{\itemsep}{0pt}\setlength{\parskip}{0pt}}
\setcounter{secnumdepth}{5}
\usepackage{booktabs}
\ifLuaTeX
  \usepackage{selnolig}  % disable illegal ligatures
\fi
\usepackage[]{natbib}
\bibliographystyle{plainnat}
\IfFileExists{bookmark.sty}{\usepackage{bookmark}}{\usepackage{hyperref}}
\IfFileExists{xurl.sty}{\usepackage{xurl}}{} % add URL line breaks if available
\urlstyle{same}
\hypersetup{
  pdftitle={stat\_helpR},
  pdfauthor={Guillaume Papuga},
  hidelinks,
  pdfcreator={LaTeX via pandoc}}

\title{stat\_helpR}
\author{Guillaume Papuga}
\date{2023-10-26}

\begin{document}
\maketitle

{
\setcounter{tocdepth}{1}
\tableofcontents
}
\hypertarget{origine-du-projet}{%
\chapter{Origine du projet}\label{origine-du-projet}}

\hypertarget{qui-suis-je}{%
\section{Qui suis je?}\label{qui-suis-je}}

\begin{verbatim}
    ◦ Botanist, plant ecologist, intêret pour l’analyse
    ◦ MCU
    ◦ Sites internet, github, etc.
\end{verbatim}

\hypertarget{pourquoi-pour-qui-uxe9crire-un-livre-sur-les-statistiques}{%
\section{Pourquoi / pour qui écrire un livre sur les statistiques?}\label{pourquoi-pour-qui-uxe9crire-un-livre-sur-les-statistiques}}

\begin{verbatim}
    ◦ Moi – futur moi (pense bete géant). J’accumule des notes ecrites (conf, etc) + scripts
    ◦ Etudiants – support de cours
    ◦ N’importe qui
    ◦ Français
\end{verbatim}

\hypertarget{pourquoi-r}{%
\section{Pourquoi R}\label{pourquoi-r}}

\begin{verbatim}
    ◦ Libre, puissant, mis à jour
    ◦ Github pour le partage
\end{verbatim}

\hypertarget{plagiat-sources}{%
\section{Plagiat \& Sources}\label{plagiat-sources}}

\begin{verbatim}
    ◦ Philo
    ◦ Livres
    ◦ Blog & internet
    ◦ Collaborateurs sur des chapitres 
\end{verbatim}

\hypertarget{report-un-bug}{%
\section{Report un bug}\label{report-un-bug}}

\begin{verbatim}
    ◦ Via github
    ◦ Mail
\end{verbatim}

\hypertarget{todolist}{%
\section{ToDoList}\label{todolist}}

\hypertarget{gestion-des-donnuxe9es}{%
\chapter{Gestion des données}\label{gestion-des-donnuxe9es}}

\hypertarget{les-donnuxe9es-sur-r}{%
\section{Les données sur R}\label{les-donnuxe9es-sur-r}}

\hypertarget{les-jeux-de-donnuxe9es-utilisuxe9s}{%
\section{Les jeux de données utilisés}\label{les-jeux-de-donnuxe9es-utilisuxe9s}}

\begin{verbatim}
    ◦ Via R
    ◦ Perso en ligne sur le repo
    ◦ JDD packages
    ◦ Comment simuler un jdd?
        Lois 
        Random 
\end{verbatim}

\hypertarget{les-fonctions-de-base}{%
\section{Les fonctions de base}\label{les-fonctions-de-base}}

\hypertarget{cruxe9er-une-fonction}{%
\section{Créer une fonction}\label{cruxe9er-une-fonction}}

\hypertarget{data-manipulation}{%
\section{Data manipulation}\label{data-manipulation}}

\begin{verbatim}
    ◦ Apply
    ◦ For
    ◦ Suite Tidy
\end{verbatim}

\hypertarget{ma-structure-danalyse-workflow}{%
\section{Ma structure d'analyse (workflow)}\label{ma-structure-danalyse-workflow}}

\hypertarget{utilisation-de-r}{%
\chapter{Utilisation de R}\label{utilisation-de-r}}

\begin{verbatim}
• Couleurs (RColorBrewer)
• Fonctionnement (parallèle sur des cœurs différents)
• GitHub
\end{verbatim}

\hypertarget{data-mining}{%
\chapter{Data mining}\label{data-mining}}

Décrire les données (pas de stat) représentation graphiques
• Package de data mining
• Matrix of scatter plot
• Scatter plot 3D et interactif
• GGplot2

\hypertarget{les-lois-statistiques}{%
\chapter{Les lois statistiques}\label{les-lois-statistiques}}

A faire

\hypertarget{test-simples}{%
\chapter{Test simples}\label{test-simples}}

\begin{verbatim}
• Tester la normalité
• Variances
• Paramétriques
\end{verbatim}

Les différentes anova

\begin{verbatim}
• Non paramétriques
• Autre, Chi2, etc. 
• Tester la corrélation entre variables
    ◦ Pearson
    ◦ Spearman
\end{verbatim}

Inférences
• bootstrap

\hypertarget{le-moduxe8le-linuxe9aire-guxe9nuxe9ralisuxe9}{%
\chapter{Le modèle linéaire généralisé}\label{le-moduxe8le-linuxe9aire-guxe9nuxe9ralisuxe9}}

Introduction
◦ Lien (quasi)
• Modèle linéaire simple (lm)
◦ Comment ecrire un modèle
• Introduction au GLM
• GLM sur données de comptage
• GLM sur loi binomiale
◦ Données présence absence
▪ Fit
▪ validation
◦ sur proportion
• Transformation des variables

Les contrastes

\hypertarget{le-moduxe8le-mixte}{%
\chapter{Le modèle mixte}\label{le-moduxe8le-mixte}}

\begin{verbatim}
• Effet mixtes (random y, random slope)
• comment ecrire le modèle
• Exemple sur loi normale
• Exemple sur loi poisson
• Exemple sur loi binomiale
\end{verbatim}

\hypertarget{statistiques-multivariuxe9es}{%
\chapter{Statistiques multivariées}\label{statistiques-multivariuxe9es}}

\begin{verbatim}
• Introduction
• Les matrices de distance
• Analyses Descriptive
    ◦ AFC 
    ◦ ACPP
    ◦ Nmds
• CAH
• KMeans
• Discriminantes
• Quelques tests
• Couplage de tableaux
\end{verbatim}

\hypertarget{times-series}{%
\chapter{Times series}\label{times-series}}

\begin{verbatim}
• Format
• Analyses de survie
\end{verbatim}

\hypertarget{analyses-spatiales}{%
\chapter{Analyses spatiales}\label{analyses-spatiales}}

\begin{verbatim}
• Représentations spatiales
    ◦ Données en ligne
    ◦ Iintégrer des données
        ▪ Raster
        ▪ vecteurs
• Operations basiques d’analyse spatiale
    ◦ Density kernel
    ◦ Spatial regression
• Détecter l’autocorrélation spatiale
\end{verbatim}

\hypertarget{sdm}{%
\chapter{SDM}\label{sdm}}

\begin{verbatim}
    ◦ Les différents algorithmes 
    ◦ Modèle averaging
• jointSDM
\end{verbatim}

\hypertarget{analyse-de-viabilituxe9-des-populations-chez-les-plantes}{%
\chapter{Analyse de viabilité des populations chez les plantes}\label{analyse-de-viabilituxe9-des-populations-chez-les-plantes}}

\begin{verbatim}
• Introduction : la question du type de données
• Matrice classique
• Lme avec structuration temporelle (données comptages)
• Lme avec structuration temporelle (données p/a)
• 
\end{verbatim}

\hypertarget{soutient-de-cours}{%
\chapter{Soutient de cours}\label{soutient-de-cours}}

\hypertarget{l2-uxe9cologie-guxe9nuxe9rale}{%
\section{L2 -- écologie générale}\label{l2-uxe9cologie-guxe9nuxe9rale}}

\hypertarget{l3-uxe9cologie-des-communautuxe9s}{%
\section{L3 -- écologie des communautés}\label{l3-uxe9cologie-des-communautuxe9s}}

\hypertarget{master}{%
\section{Master}\label{master}}

\hypertarget{references}{%
\chapter{References}\label{references}}

  \bibliography{book.bib,packages.bib}

\end{document}
